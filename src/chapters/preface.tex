\chapter*{Preface}
\addcontentsline{toc}{chapter}{Preface}

I originally set out to write this book because I (like many others in academia)
felt like the mathematics educational system has critically underserved far too
many individuals. I do concede that the landscape of the dynamics of individuals'
brains is far too complex and expansive for there to be any reasonable expectation
that all should obtain a mastery (or even perhaps a comfort) with mathematical
reasoning. But should there be even exist a modest threshold that we can expect a
generous majority of individuals to reach, we have still fallen far short of it.
And so I want to address that knowledge gap, focusing on adults who have felt
mistreated by the math education system.

In doing so, there will be several challenges: first and foremost, it is generally
accepted that the child brain is more plastic than that of the adult. Children
tend to be more creative and accepting of new ideas than adults. Secondly, adult
learners tend to bring previous ``baggage'' into the picture when learning, whether
in the form of faulty reasoning they have already internalized, or cognitive
biases from previous negative learning experiences. But the adult brain also has
its advantages: adults boast better abstract and critical thinking abilities,
and a wider range of life experiences that can be useful for appealing to intuition.

As a result of these (among other) differences, it is clear that we have to adopt
a different learning approach than the standard pedagogical methods. This will be
a major theme throughout the text, and one that we plan to employ to great success.
Hopefully, we both learn something along the way.

\flushright
\textit{-SP}
\flushleft

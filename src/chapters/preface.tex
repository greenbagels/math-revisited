\chapter*{Preface}
\addcontentsline{toc}{chapter}{Preface}

It is the unfortunate case that, at least in the United States, math education
largely fails to supply students with anything but the bare minimum (if that)
skills necessary for them to achieve proper \textit{mathematical literacy}.
And while such individuals likely develop a stronger sense of abstract reasoning
as adults (compared to their youth), they are also more prone to overthinking
concepts and allowing their preconceptions to obstruct their ability to embrace
new and unfamiliar (or perhaps more aptly, \textit{uncomfortable}) ideas.

By virtue of educational systems extending beyond the confines of the classroom,
being intertwined with the social and political circumstances of its participants,
there is little hope of pinning the blame for such a phenomenon on a single
culprit, such as educational materials.
In light of this (among other details), my goal in writing this text is not to
single-handedly ``solve'' or ``revolutionize'' math education.
Instead, I merely aim to gather my thoughts and conclusions from years of both
studying and teaching college and pre-college mathematics, and provide a personal
take on effective exposition of fundamental mathematical concepts.
However, in the grand scheme of mathematics, I am but a lowly acolyte working
along; so while I offer no guarantees as to the usefulness of my writing, I hope
that it may serve as some help to those that have previously felt abandoned or
intimidated in their mathematical studies.

\flushright
\textit{-SP}
\flushleft

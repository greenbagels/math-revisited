\chapter{Numbers}

A common point of contention between teachers, parents, and students in US
schools concerns the real life utility of more advanced secondary school math
topics such as trigonometry and calculus. As an example, let's take a look at
a problem from James Stewart's \textit{Calculus: Early Transcendentals} that
discusses the standard calculus technique of \textit{related rates}. (We will
\textit{not} try to solve this problem, as it is far beyond the scope of our
current goals, so do not feel overwhelmed):

``A water tank has the shape of an inverted circular cone with base radius 2 m
and height 4 m. If water is being pumped into the tank at a rate of 2 m$^3$/min,
find the rate at which the water level is rising when the water is 3 m deep.''

The point of the problem is that the height of the water in the tank over time
has some non-obvious relationship with the actual total amount of water in the
tank at any given time. With calculus, it's fairly straightforward to find this
relationship exactly (and thus predict how the water level changes over time).
But, unless you're an engineer dealing with, say, industrial plants where water
tanks are constantly being filled and drained, when would you need to know how
to do this? Maybe it's reasonable after all that parents across the country
voice frustration that their children are required to learn this if they want
to become a carpenter.

But what happens if we consider something much more fundamental, like
\textbf{arithmetic}, the study of numbers and common ways we can combine numbers?
While it's probably unexpected that someone would object to the idea that
arithmetic is a useful skill to master, let's go ahead and think about why this
is the case anyway.

Consider your morning routine on an average day, in an average week.
Now, try to write out or narrate such a routine without using any
\textbf{quantitative} language. No words that describe concepts like ``how many''.
It's certainly possible that we can accomplish this task in language that
resembles a somewhat basic form of English:

\[
    \textit{I ate orange.}
\]

In this case, we intend ``orange'' to mean some indeterminate quantity of fruit.
It could mean one, two, or more, but we don't care to convey this detail at all.
Naturally, this poses a problem: we have lost any way to tell if some person eats
a single orange on a given day, or if they are at the will of some cruel villain
who forces them to consume a dozen oranges a day or suffer dire consequences.
Now, when providing this motivation for introducing the concept of numbers, a
skeptical student might suggest that we are simply number shills paid off by Big
Number, and that there must exist some middle ground solution to our problematic
scenario.
What if we don't need the ability to describe exactly how much of something
exists --- maybe it suffices to be able to tell whether there is None of
something, Some of something, and maybe even Many of something.

